% Options for packages loaded elsewhere
\PassOptionsToPackage{unicode}{hyperref}
\PassOptionsToPackage{hyphens}{url}
%
\documentclass[
  12,
  table]{proadi}
\usepackage{lmodern}
\usepackage{amssymb,amsmath}
\usepackage{ifxetex,ifluatex}
\ifnum 0\ifxetex 1\fi\ifluatex 1\fi=0 % if pdftex
  \usepackage[T1]{fontenc}
  \usepackage[utf8]{inputenc}
  \usepackage{textcomp} % provide euro and other symbols
\else % if luatex or xetex
  \usepackage{unicode-math}
  \defaultfontfeatures{Scale=MatchLowercase}
  \defaultfontfeatures[\rmfamily]{Ligatures=TeX,Scale=1}
\fi
% Use upquote if available, for straight quotes in verbatim environments
\IfFileExists{upquote.sty}{\usepackage{upquote}}{}
\IfFileExists{microtype.sty}{% use microtype if available
  \usepackage[]{microtype}
  \UseMicrotypeSet[protrusion]{basicmath} % disable protrusion for tt fonts
}{}
\makeatletter
\@ifundefined{KOMAClassName}{% if non-KOMA class
  \IfFileExists{parskip.sty}{%
    \usepackage{parskip}
  }{% else
    \setlength{\parindent}{0pt}
    \setlength{\parskip}{6pt plus 2pt minus 1pt}}
}{% if KOMA class
  \KOMAoptions{parskip=half}}
\makeatother
\usepackage{xcolor}
\IfFileExists{xurl.sty}{\usepackage{xurl}}{} % add URL line breaks if available
\IfFileExists{bookmark.sty}{\usepackage{bookmark}}{\usepackage{hyperref}}
\hypersetup{
  hidelinks,
  pdfcreator={LaTeX via pandoc}}
\urlstyle{same} % disable monospaced font for URLs
\usepackage{graphicx,grffile}
\makeatletter
\def\maxwidth{\ifdim\Gin@nat@width>\linewidth\linewidth\else\Gin@nat@width\fi}
\def\maxheight{\ifdim\Gin@nat@height>\textheight\textheight\else\Gin@nat@height\fi}
\makeatother
% Scale images if necessary, so that they will not overflow the page
% margins by default, and it is still possible to overwrite the defaults
% using explicit options in \includegraphics[width, height, ...]{}
\setkeys{Gin}{width=\maxwidth,height=\maxheight,keepaspectratio}
% Set default figure placement to htbp
\makeatletter
\def\fps@figure{htbp}
\makeatother
\setlength{\emergencystretch}{3em} % prevent overfull lines
\providecommand{\tightlist}{%
  \setlength{\itemsep}{0pt}\setlength{\parskip}{0pt}}
\setcounter{secnumdepth}{5}
\base{sim}
\usepackage[utf8]{inputenc}
\usepackage[brazil]{babel}
\usepackage{longtable,booktabs}
\usepackage{pdflscape}
\usepackage{caption}
\usepackage{lscape}
\usepackage{xcolor}
\hypersetup{draft}

\author{}
\date{\vspace{-2.5em}}

\begin{document}

\hypertarget{qualidade-de-dados}{%
\section{Qualidade de dados}\label{qualidade-de-dados}}

O processo de análise de qualidade de dados está focado na avaliação de
conjuntos de dados e na aplicação de ações corretivas, para garantir que
estes estejam adequados aos propósitos para os quais foram originalmente
destinados (Merino et al. 2016). Dessa forma, a qualidade de dados está
diretamente relacionada a confiabilidade dos dados de entrada.
Considerando que os dados têm níveis inadequados de qualidade, é
provável que ocorram erros, que podem se propagar acidentalmente e
inconscientemente por todo o fluxo da informação, prejudicando a
eficiência do sistema. Formas regulares de avaliar a qualidade de dados
com modelos clássicos geralmente se destinam a detectar e corrigir erros
em fontes conhecidas com base em um conjunto limitado de regras. No
ambiente de \emph{Big Data}, a quantidade de regras pode ser enorme e o
custo da aplicação para correção de erros pode não ser viável e nem
apropriado. Isso ocorre principalmente porque o \emph{Big Data} não é
apenas sobre dados, mas também sobre uma pilha conceitual e tecnológica
completa, incluindo dados brutos e processados, armazenamento, formas de
gerenciar dados, processamento e análise (Merino et al. 2016).

Uma dimensão de qualidade de dados é um termo descritor de um recurso de
dados, o qual pode ser medido ou avaliado de acordo com padrões
definidos, a fim de determinar a qualidade de um conjunto de dados.
Geralmente, dados só têm valor quando dão suporte a um processo ou a uma
tomada de decisão. Em consequência, as regras de qualidade de dados
definidas devem levar em consideração o valor que os dados podem
fornecer para o sistema. Nesse contexto, as seguintes dimensões de
qualidade de dados são analisadas: Completude, Conformidade, Acurácia,
Consistência, Unicidade, Temporalidade

\textbf{Completude} caracteriza a taxa de preenchimento das variáveis.
Para cada variável é calculado o percentual de entradas com informação
não nulas, respeitando, quando houver, sua dependência com outras
variáveis.

\textbf{Conformidade} detecta concordância nos valores digitados nos
campos das variáveis, avaliando se os valores de entrada não nulos estão
em conformidade com os padrões descritos pelo dicionário de dados. Para
cada variável estudada é calculado o percentual de entradas em
conformidade com o padrão adotado.

\textbf{Acurácia} visa detectar se informação registrada reflete o
evento ou objeto descrito, isto é, verificar se o dado cadastrado está
em concordância com o evento observado. Devido ao processo de
anonimização dos dados, a análise de acurácia se restringe a verificar a
possibilidade das informações registradas. Note que acurácia e
conformidade são dimensões distintas, pois enquanto conformidade avalia
o padrão do dado, acurácia avalia a razoabilidade dos dados. Para cada
variável estudada é calculado o percentual de entradas com informações
acuradas.

\textbf{Consistência} constitui de testes envolvendo duas ou mais
variáveis visando detectar inconsistências entre dados de um mesmo
registro. Para cada teste considerado é calculado os percentuais de
aprovação e falha.

\textbf{Unicidade} objetiva mensurar o grau de duplicidade nos dados,
realizando a busca por meio de identificadores dos pacientes.

\textbf{Temporalidade} objetiva efetuar medidas estatísticas nos
intervalos de tempos entre eventos, por exemplo, o nascimento de um
recém-nascido e inclusão desse registro no sistema. O principal
interesse é verificar se o dado é disponibilizado prontamente.

\renewcommand{\arraystretch}{1.25}

\renewcommand{\arraystretch}{1}

\hypertarget{muxe9todos}{%
\section{Métodos}\label{muxe9todos}}

A análise apresentada constitui-se de um esquema cíclico, iniciando no
mapeamento da documentação e do comportamento dos dados, através da
observação de trechos das bases. Em seguida, são definidas as variáveis
de teste. Após, ocorre a obtenção e avaliação dos resultados obtidos,
recorrendo, e se necessário retificando, conclusões obtidas nos passos
anteriores. Nesse contexto, são definidos parâmetros a serem passados
para as funções relativas às metricas citadas e implementação de
\emph{queries} para os testes de consistência, sintetizados em um único
\emph{script} relativo à base analisada.

O maneio dos dados ocorreu através dos serviços \textbf{Amazon Athena} e
\textbf{Amazon S3}, assim como testes e análises se deu utilizando
\textbf{linguagem R}. Os \emph{scripts} utilizados estão disponíveis em
um repositório de qualidade de dados no \emph{GitHub}. Enfatiza-se que
esses dados podem sofrer alterações, caso ocorram atualizações.

O dicionário de dado utilizado, não oficial, é inferido das descrições
das variáveis contidas nos relatórios de integração e é apresentado
neste relatório. No que tange os testes de unicidade, procurou-se
analisar apenas as informações individuais dos pacientes.

Mudanças no domínio e tamanho de caracteres das variáveis são
detectadas, relatadas e consideradas no cálculo de medidas de qualidade
dos dados.

O Cômputo dos resultados numéricos ocorre de modo cascata, isto é, os
registros submetidos ao teste de conformidade devem ser não nulos, os
registros submetidos ao teste de acurácia devem estar conformes, os
registros submetidos aos testes de consistência devem estar acurados, e
quando não for possível, conformes, sendo que o mesmo se aplica aos
registros submetidos aos testes de unicidade. Em prosseguimento, os
resultados numéricos são avaliados nas dimensões analisadas
calculando-se a média ponderada dos testes realizados, utilizando como
peso o total de registros por variável. Para a consistência, é realizado
um ajuste em que todas as variáveis testadas devem existir
simultaneamente.

Objetivando avaliar a base de dados, o conjunto de resultados
representando cada dimensão foi classificada como excelente
(\textgreater{} 90\%), ótimo (75\% - 89,9\%), regular (50\% - 74,9\%) ou
ruim (\textless{} 49,9\%), baseado nos relatórios do livro \emph{Saúde
Brasil}, organizado pela Secretaria de Vigilância em Saúde (Brasil.
Ministério da Saúde. Secretaria de Vigilância em Saúde. Departamento de
Análise de Situação em Saúde 2019). Em decorrência do método cascata
utilizado, é realizado o produto dos resultados obtidos, caracterizando
a qualidade da base de dados como um todo, que também pode ser
classificada considerando as classes definidas em \emph{Saúde Brasil}
(Brasil. Ministério da Saúde. Secretaria de Vigilância em Saúde.
Departamento de Análise de Situação em Saúde 2019).

\hypertarget{disponibilidade-dos-dados}{%
\section{Disponibilidade dos dados}\label{disponibilidade-dos-dados}}

Esta análise têm o objetivo de dissertar acerca da disponibilidade dos
dados em todo o período representado pela base de dados e em todas as
Unidades Federativas.

Após realização de testes, averiguou-se que para todo o período
representado pela base de dados a informação está disponível, ou seja,
existem registros relativos a todos os anos, meses e Unidades
Federativas.

\hypertarget{variuxe1veis-existentes-e-mudanuxe7as-ocorridas}{%
\section{Variáveis existentes e mudanças
ocorridas}\label{variuxe1veis-existentes-e-mudanuxe7as-ocorridas}}

Esta análise tem o objetivo identificar as variáveis existentes na base
de dado se relatar as mudanças ocorridas ao longo do tempo.

Após realização de testes não identificou-se qualquer alteração
significante nas variáveis.

\hypertarget{resultados}{%
\section{Resultados}\label{resultados}}

Descrições das variáveis são apresentadas no
\protect\hyperlink{dicionuxe1rio-adotado}{Dicionário adotado}.
Resultados dos testes de completude, conformidade e acurácia são
exibidos nos \protect\hyperlink{resultados-numuxe9ricos}{Resultados
numéricos}, onde estão organizados em duas tabelas: resultado geral e
resultado agregado por ano. Uma descrição mais detalhada dos testes de
inconsistência realizados, bem como seus respectivos resultados
numéricos estão descritos em
\protect\hyperlink{testes-de-inconsistuxeancia}{Testes de
inconsistência}.

\hypertarget{completude}{%
\subsection{Completude}\label{completude}}

Nesta dimensão são detectados valores faltantes através da busca pelas
constantes representando valores ausentes. Nesse sentido, considerou-se
como incompletos os registros contendo os valores \texttt{NA} constante
lógica que indica valor ausente, e \texttt{NULL}, que representa objetos
nulos.

No geral, os resultados de completude das variáveis estão distribuídas
pelas categorias definidas em \protect\hyperlink{muxe9todos}{Métodos}
segundo o gráfico a seguir. O resultado percentual por variável está
descrito nos \protect\hyperlink{resultados-numuxe9ricos}{Resultados
numéricos}. O cômputo da média ponderada dos resultados obtidos é de
\textbf{28.59\%}, ou seja, a \textbf{completude é ruim}.

\begin{figure}
\centering
\includegraphics[width=0.8\textwidth,height=\textheight]{imagens/comp.png}
\caption{distribuição dos resultados de completude.}
\end{figure}

\hypertarget{conformidade}{%
\subsection{Conformidade}\label{conformidade}}

Verificou-se se os dados apresentam os padrões descritos no
\protect\hyperlink{dicionuxe1rio-adotado}{dicionário de dados adotado}
como referência a respeito da quantidade de caracteres e valores
válidos.

Ressalta-se que durante a construção do dicionário de dados não foi
possível obter os microdados ou informação equivalente contendo, quando
existente, os valores válidos de domínio para algumas variáveis. Nesse
sentido, a tabela a seguir apresenta ao máximo dez registros mais
frequentes para essas variáveis, separados por vírgulas.

\begingroup\fontsize{10}{12}\selectfont

\begin{longtable}[t]{>{}l>{\raggedright\arraybackslash}p{10cm}}
\caption{\label{tab:unnamed-chunk-12}registros mais frequentes por variável em que não foi possível obter a descrição.}\\
\toprule
Variável & Domínio\\
\midrule
\em{causamat} & O937\\
\em{cb\_orig} & A49, B57, C17, C25, C71, E10, E11, E14, F02, I21\\
\em{codbaiocor} & 0.0, 1.0, 2.0, 27020.0, 27191.0, 3.0, 4.0, 5.0, 6.0, 7.0\\
\em{codbaires} & 0.0, 1.0, 2.0, 27020.0, 2719.0, 27191.0, 3.0, 4.0, 5.0, 6.0\\
\em{codcart} & , 12744.0, 1327.0, 1449.0, 2044.0, 2095.0, 2560.0, 2561.0, 2706.0, 2725.0\\
\addlinespace
\em{codestcart} & 27.0\\
\em{codestocor} & 27.0\\
\em{codestres} & 15.0, 21.0, 23.0, 24.0, 25.0, 26.0, 27.0, 28.0, 29.0, 33.0\\
\em{codmuncart} & , 270010.0, 270030.0, 270050.0, 270060.0, 270080.0, 270090.0, 270120.0, 270135.0, 270150.0\\
\em{codmunnatu} & 270030.0, 270640.0, 270710.0, 270730.0, 270930.0, 280030.0,\\
\addlinespace
\em{codpaisres} & 1.0, 107.0, 32.0, 3.0\\
\em{codregocor} & 2.0\\
\em{codregres} & 1.0, 2.0, 3.0, 4.0, 5.0\\
\em{comunsvoim} & 270030.0, 270430.0, 261160.0, 270330.0, 270470.0, 270730.0, 270100.0, 270300.0, 270770.0, 270800.0\\
\em{cond\_mat} & 1.0, 51.0, 60.0, 88.0\\
\addlinespace
\em{confcausa} & F\\
\em{confidade} & I\\
\em{critica} & N, S\\
\em{dstempo} & / / / / /,  / / /1 0/,  / /2 1/ /,  /11 1/ /,  /5 0/ / /, 0 4/ / / /, 0 4/ /0 4/, 0 4/0 4/ /, 0 4/0 4/0, 01 2/ / /\\
\em{dtencomite} & 2010-02-22, 2010-05-18, 2010-06-21, 2010-08-10, 2010-11-09, 2010-11-11, 2010-12-07, 2011-01-18, 2011-02-14\\
\addlinespace
\em{dtressele} & 2008-11-04, 2008-11-05, 2008-11-06, 2009-06-24, 2009-06-25, 2009-06-26, 2010-09-23, 2010-09-24, 2010-10-07, 2011-08-16\\
\em{fontes} & SSSSXX, SSSXSX, SSSXXS, SSSXXX, SSXXXS, SSXXXX, SXSXXS, SXSXXX, SXXXSS, SXXXSX\\
\em{fontesinf} & SSSSXXS, SSSSXXX, SSXSXXX, SSXXXXX, SXSSXSX, SXSSXXS, SXSSXXX, SXSXXXX, SXXSSXX, SXXSXSX\\
\em{linhaii} & *,\\
\em{natural\_} & , 10.0, 125.0, 144.0, 186.0, 19.0, 190.0, 31.0, 74.0, 77.0\\
\addlinespace
\em{nressele} & [02] - NÒo processado devido a regra causa externa., [03] - NÒo processado devido a regra de procedimento mÚdico., [04] - NÒo processado devido a regra causa paralisia, [04] - NÒo processado devido a regra causa paralisia[04] - NÒo process, [05] - NÒo executado devido a Regra F, [06] - NÒo executado devido ao codigo I663 nÒo poder ser CB. Tem a nec, [06] - NÒo executado devido ao codigo R69  nÒo poder ser CB. Tem a nec\\
\em{numerodv} & 0.0, 1.0, 2.0, 3.0, 4.0, 5.0, 6.0, 7.0, 8.0, 9.0\\
\em{origem} & 1.0, 2.0\\
\em{qtdfilmort} & 0.0, 1.0, 11.0, 2.0, 3.0, 4.0, 5.0, 8.0, 9.0, 99.0\\
\em{qtdfilvivo} & 0.0, 1.0, 10.0, 11.0, 12.0, 13.0, 14.0, 15.0, 18.0, 2.0\\
\addlinespace
\em{semagestac} & 1.0, 24.0, 27.0, 32.0, 33.0, 34.0, 38.0, 4.0, 40.0, 41.0\\
\em{ufinform} & 27.0, 29.0, 26.0\\
\em{vrsressele} & 2.0\\
\bottomrule
\end{longtable}
\endgroup{}

Acerca das inconformidades identificadas, a tabela a seguir apresenta os
registros mais frequentes, limitado a cinco registros, que encontram-se
nesta situação.

\begingroup\fontsize{10}{12}\selectfont

\begin{longtable}[t]{>{}l>{\raggedright\arraybackslash}p{10cm}}
\caption{\label{tab:unnamed-chunk-13}registros inconformes por variável.}\\
\toprule
Variável & Registros inconformes\\
\midrule
\em{ESC} & 0,\textbackslash{}sNA\\
\em{ESCMAE} & 0,\textbackslash{}sNA\\
\bottomrule
\end{longtable}
\endgroup{}

No geral, os resultados de conformidade das variáveis estão distribuídas
pelas categorias definidas em \protect\hyperlink{muxe9todos}{Métodos}
segundo o gráfico a seguir. O resultado percentual por variável está
descrito nos \protect\hyperlink{resultados-numuxe9ricos}{Resultados
numéricos}. O cômputo da média ponderada dos resultados obtidos é de
\textbf{96.49\%}, ou seja, a \textbf{conformidade é excelente}.

\begin{figure}
\centering
\includegraphics[width=0.8\textwidth,height=\textheight]{imagens/conf.png}
\caption{distribuição dos resultados de conformidade.}
\end{figure}

\hypertarget{acuruxe1cia}{%
\subsection{Acurácia}\label{acuruxe1cia}}

Inicialmente a métrica de acurácia foi aplicada a dois tipos de
registros: datas, ao verificar se o dado configura-se uma data válida e
condizente ao período representado pela base de dados; nomes e códigos
de municípios, ao verificar se estão contidos na tabela de códigos de
municípios e estados do IBGE\footnote{\url{https://www.ibge.gov.br/explica/codigos-dos-municipios.php}}.
Após esta análise de datas e municípios é realizada investigação acerca
do preenchimento das variáveis com o objetivo de detectar a presença de
preenchimentos sem informações relevantes. Em seguida, são verificados
os registros representando informações numéricas, a respeito do sinal
(\emph{e.g.} número de filhos deve ser positivo) e do conjunto ao qual
pertence (\emph{e.g.} número de filhos deve ser um número inteiro).

No que tange propriamente o preenchimento dos registros, buscou-se
identificar valores representando a ausência de informações ou que foram
ignorados, além de sequências finitas do caractere espaço
(\emph{whitespace}) ou sequências finitas do numeral zero para variáveis
não numéricas. Este fato pode representar um problema, visto que estará
de acordo ao tamanho estabelecido pelo dicionário de dados, porém não
estará acurado, não representando informação alguma. Para realizar a
identificação no primeiro caso, utilizou-se o método TF-IDF.

O método TF-IDF\footnote{\url{https://bergvca.github.io/2017/10/14/super-fast-string-matching.html}},
em conjunto aos métodos \emph{N-Grams} e multiplicação de matriz
esparsa, é uma medida estatística que tem o intuito de indicar a
similaridade de uma palavra em relação a outra. TF-IDF é um método para
gerar recursos do texto multiplicando a frequência de um termo em um
documento (\emph{Term Frequency}, ou TF) pela importância (\emph{Inverse
Document Frequency}, ou IDF) do mesmo termo em um corpus inteiro. Este
método é muito útil na classificação e no agrupamento de textos e é
usado para transformar documentos em vetores numéricos, que podem ser
facilmente comparados. Embora os termos no TF-IDF sejam geralmente
palavras, essa condição não é necessária. Como a maioria dos registros
possuem de uma a três palavras, utilizou-se \emph{N-Grams}: sequências
de \emph{N} caracteres contíguos. Para avaliação, calculou-se a
proximidade dos vetores resultantes do método TF-IDF, através da
semelhança cosseno, que pode ser vista como um produto escalar
normalizado.

Após aplicação do método, não identificou-se qualquer registro nessa
situação.

Em relação as várias quantitativas, a tabela a seguir expõe os valores
atípicos detectados\footnote{\url{https://www.rdocumentation.org/packages/grDevices/versions/3.6.2/topics/boxplot.stats}},
isto é, registros numéricos que apresentam grande afastamento em relação
aos demais, dentro do universo de uma única variável, os quais implicam,
tipicamente, em prejuízos a interpretação dos resultados dos testes
estatísticos aplicados.

\begingroup\fontsize{10}{12}\selectfont

\begin{longtable}[t]{>{}l>{\raggedright\arraybackslash}p{10cm}}
\caption{\label{tab:unnamed-chunk-15}valores atípicos para variáveis numéricas.}\\
\toprule
Variável & Valores atípicos\\
\midrule
\endfirsthead
\caption[]{valores atípicos para variáveis numéricas. \textit{(continued)}}\\
\toprule
Variável & Valores atípicos\\
\midrule
\endhead

\endfoot
\bottomrule
\endlastfoot
\em{difdata} & 156, 157, 158, 159, 160, 161, 162, 163, 164, 165, ..., 866, 868, 874, 878, 881, 883, 884, 885, 887, 914\\
\em{idademae} & 45, 46, 48, 49, 52, 54, 60, 99, ..., 45, 46, 48, 49, 52, 54, 60, 99\\
\em{nudiasobco} & 231, 232, 233, 235, 236, 237, 238, 239, 240, 241, ..., 438, 442, 445, 457, 458, 505, 507, 528, 550, 594\\
\em{peso} & 5965, 6050, 6557, 6655, ..., 5965, 6050, 6557, 6655\\
\em{qtdfilmort} & 3, 4, 5, 6, 7, 8, 9, 10, 11, 20, ..., 6, 7, 8, 9, 10, 11, 20, 21, 30, 99\\
\addlinespace
\em{qtdfilvivo} & 6, 7, 8, 9, 10, 11, 12, 13, 14, 15, ..., 13, 14, 15, 17, 18, 19, 20, 21, 25, 99\\
\em{semagestac} & 0, 1, 2, 3, 4, 5, 6, 7, 8, 9, ..., 2, 3, 4, 5, 6, 7, 8, 9, 10, 99\\*
\end{longtable}
\endgroup{}

Acerca das inacuracidades identificadas, a tabela a seguir apresenta os
registros mais frequentes, limitado a cinco registros, que encontram-se
nesta situação.

\begingroup\fontsize{10}{12}\selectfont

\begin{longtable}[t]{>{}l>{\raggedright\arraybackslash}p{10cm}}
\caption{\label{tab:unnamed-chunk-16}registros inacurados por variável.}\\
\toprule
Variável & Registros inacurados\\
\midrule
\em{ESC} & 9\\
\em{ESCMAE} & 9\\
\bottomrule
\end{longtable}
\endgroup{}

No geral, os resultados de acurácia das variáveis estão distribuídas
pelas categorias definidas em \protect\hyperlink{muxe9todos}{Métodos}
segundo o gráfico a seguir. O resultado percentual por variável está
descrito nos \protect\hyperlink{resultados-numuxe9ricos}{Resultados
numéricos}. O cômputo da média ponderada dos resultados obtidos é de
\textbf{98.18\%}, ou seja, a \textbf{acurácia é excelente}.

\begin{figure}
\centering
\includegraphics[width=0.8\textwidth,height=\textheight]{imagens/ac.png}
\caption{distribuição dos resultados de acurácia.}
\end{figure}

\hypertarget{consistuxeancia}{%
\subsection{Consistência}\label{consistuxeancia}}

Os resultados apresentados são de testes aplicados a um mesmo registro,
ou seja, mesma linha do conjunto de dados. Estes testes detectam
principalmente problemas na entrada de dados envolvendo condições
específicas de inconsistências. Uma descrição mais detalhada de cada
teste está presente em
\protect\hyperlink{testes-de-inconsistuxeancia}{Testes de
inconsistência}.

\begin{table}[H]

\caption{\label{tab:unnamed-chunk-17}resultados de consistência.}
\centering
\fontsize{10}{12}\selectfont
\begin{tabular}[t]{>{\centering\arraybackslash}p{1cm}>{\raggedright\arraybackslash}p{9cm}>{\centering\arraybackslash}p{3cm}}
\toprule
Teste & Descrição & Falhas [partes por mil]\\
\midrule
\em{T1} & dtobito > dtrecorig & 0.561\\
\em{T2} & dtobito > dtcadastro & 0.463\\
\em{T3} & obitopuerp == 1|2|3 \& sexo == 1 & 0.442\\
\em{T4} & dtobito > dtrecoriga & 0.337\\
\em{T5} & dtobito > dtregcart & 0.106\\
\addlinespace
\em{T6} & dtobito > dtatestado & 0.089\\
\em{T7} & dtobito > dtrecebim & 0.055\\
\em{T8} & dtobito > dtinvestig & 0.047\\
\em{T9} & codestab !=null \& lococor != 1\&2 & 0.007\\
\em{T10} & obitoparto == 2 \& dtnasc != dtobito & 0.000\\
\addlinespace
\em{T11} & obitoparto == 1 \& dtnasc < dtobito & 0.000\\
\em{T12} & obitograv == 1 \& sexo == 1 & 0.000\\
\em{T13} & comunsvoim != null \& atestante != 3\&4 & 0.000\\
\em{T14} & esc > idade & 0.000\\
\em{T15} & dtobito < dtnasc & 0.000\\
\addlinespace
\em{T16} & escmae > idademae & 0.000\\
\em{T17} & tipobito == 1 \& dtnasc != dtobito & 0.000\\
\em{T18} & tppos == ‘n’ \& dtinvestig != null & 0.000\\
\bottomrule
\end{tabular}
\end{table}

A distribuição temporal dos resultados dos testes de consistência é
apresentada no gráfico a seguir

\begin{figure}
\centering
\includegraphics[width=0.8\textwidth,height=\textheight]{imagens/cons-anual.png}
\caption{distribuição temporal da consistência.}
\end{figure}

O resultado de cada teste por ano está descrito em
\protect\hyperlink{testes-de-inconsistuxeancia}{Testes de
inconsistência}. O cômputo da média ponderada dos resultados obtidos é
de \textbf{99.99\%}, ou seja, a \textbf{consistência é excelente}.

\hypertarget{unicidade}{%
\subsection{Unicidade}\label{unicidade}}

\label{sub:unicidade}

Nesta dimensão é calculado o grau de duplicidade dos dados, buscando
diferenças por meio dos identificadores dos pacientes.

São excluídos deste teste os identificadores de pacientes presentes
apenas uma vez na base de dados. Nesse sentido, é apresentado a seguir a
distribuição de frequência desta variável. A saber, a quantidade
associada a categoria \emph{ID = 1} representa o percentual de pacientes
com apenas 1 registro no sistema, enquanto a quantidade associada a a
categoria \emph{ID = 2} representa o percentual de pacientes com 2
registros. Mesma interpretação pode ser realizada para as demais
categorias.

\begin{table}[H]

\caption{\label{tab:unnamed-chunk-18}frequência de identificadores. A quantidade associada a categoria \textit{ID = 1} representa o percentual de pacientes com apenas 1 registro no sistema, enquanto a quantidade associada a a categoria \textit{ID = 2} representa o percentual de pacientes com 2 registros. Mesma interpretação pode ser realizada para as demais categorias.}
\centering
\fontsize{10}{12}\selectfont
\begin{tabular}[t]{>{}lc}
\toprule
Categoria & Frequência [\%]\\
\midrule
\em{ID = 1} & 99.88\\
\em{ID = 2} & 0.12\\
\em{ID = 3} & 0.00\\
\em{ID = 4} & 0.00\\
\em{5 <= ID < 10} & 0.00\\
\addlinespace
\em{10 <= ID < 50} & 0.00\\
\em{50 <= ID < 100} & 0.00\\
\em{ID >= 100} & 0.00\\
\bottomrule
\end{tabular}
\end{table}

\begin{table}[H]

\caption{\label{tab:unnamed-chunk-19}resultados de unicidade por variável relacionada à identificação do paciente.}
\centering
\fontsize{10}{12}\selectfont
\begin{tabular}[t]{>{}c>{}lc}
\toprule
Teste & Variável & Unicidade [\%]\\
\midrule
\em{T1} & \em{dtnasc} & 81.94\\
\em{T2} & \em{numerodo} & 37.44\\
\em{T3} & \em{horaobito} & 46.70\\
\em{T4} & \em{sexo} & 98.68\\
\em{T5} & \em{lococor} & 64.76\\
\addlinespace
\em{T6} & \em{idade} & 65.64\\
\em{T7} & \em{racacor} & 76.21\\
\em{T8} & \em{codpaisres} & 100.00\\
\em{T9} & \em{dtobito} & 44.93\\
\bottomrule
\end{tabular}
\end{table}

A distribuição temporal dos resultados dos testes de unicidade é
apresentada no gráfico a seguir.

\begin{figure}
\centering
\includegraphics[width=0.8\textwidth,height=\textheight]{imagens/unic-anual.png}
\caption{distribuição temporal da unicidade.}
\end{figure}

O cômputo da média ponderada dos resultados obtidos é de
\textbf{35.68\%}, ou seja, a \textbf{unicidade é ruim}.

\hypertarget{temporalidade}{%
\subsection{Temporalidade}\label{temporalidade}}

Para mensurar esta dimensão é calculada a quantidade de dias entre duas
variáveis representando datas que estejam conformes, acuradas e
consistentes.

\begin{table}[H]

\caption{\label{tab:unnamed-chunk-20}resultados de temporalidade.}
\centering
\fontsize{10}{12}\selectfont
\begin{tabular}[t]{>{}c>{}l>{}lccc}
\toprule
Teste & Variável Inicial & Variável final & Mediana & Min. & Max.\\
\midrule
\em{T1} & \em{dtobito} & \em{dtcadastro} & 24 & 0 & 900\\
\em{T2} & \em{dtobito} & \em{dtinvestig} & 41 & 0 & 910\\
\em{T3} & \em{dtobito} & \em{dtrecebim} & 39 & 0 & 950\\
\bottomrule
\end{tabular}
\end{table}

A distribuição temporal das medianas obtidas pelos testes de
temporalidade é apresentada no gráfico a seguir.

\begin{figure}
\centering
\includegraphics[width=0.8\textwidth,height=\textheight]{imagens/temp-anual.png}
\caption{distribuição temporal da temporalidade.}
\end{figure}

\hypertarget{considerauxe7uxf5es-finais}{%
\section{Considerações finais}\label{considerauxe7uxf5es-finais}}

A avaliação realizada é especialmente oportuna, tendo em vista o cenário
nacional e o atual empenho em fomentar o debate em torno da qualidade
das informações acerca de estabelecimentos de saúde do país.

Assim, a média ponderada dos resultados de Completude é 28.59\%, de
Conformidade é 96.49\%, de Acurácia é 98.18\%, de Consistência é
99.99\%, de Unicidade é 35.68\%. Realizando o produto destes resultados,
obtêm-se \textbf{9.66\%}, caracterizando a \textbf{base de dados como
ruim}.

\newpage

\hypertarget{referuxeancias}{%
\section{Referências}\label{referuxeancias}}

\hypertarget{refs}{}
\leavevmode\hypertarget{ref-brasil2019}{}%
Brasil. Ministério da Saúde. Secretaria de Vigilância em Saúde.
Departamento de Análise de Situação em Saúde. 2019. \emph{Saúde Brasil
2019: Uma Análise Da Situação de Saúde Com Enfoque Nas Doenças
Imunopreveníveis E Na Imunização}. Ministério da Saúde.

\leavevmode\hypertarget{ref-merino}{}%
Merino, Jorge, Ismael Caballero, Bibiano Rivas, Manuel Serrano, and
Mario Piattini. 2016. ``A Data Quality in Use Model for Big Data.''
\emph{Future Generation Computer Systems} 63: 123--30.
\url{https://doi.org/https://doi.org/10.1016/j.future.2015.11.024}.

\captionsetup[table]{labelformat=empty}

\newpage

\hypertarget{dicionuxe1rio-adotado}{%
\section*{Dicionário adotado}\label{dicionuxe1rio-adotado}}
\addcontentsline{toc}{section}{Dicionário adotado}

\begingroup\fontsize{10}{12}\selectfont

\begin{longtable}{>{}l>{\raggedright\arraybackslash}p{9cm}>{\centering\arraybackslash}p{2cm}}
\toprule
Variável & Descrição & Tamanho\\
\midrule
\endfirsthead
\multicolumn{3}{@{}l}{\textit{(continued)}}\\
\toprule
Variável & Descrição & Tamanho\\
\midrule
\endhead

\endfoot
\bottomrule
\endlastfoot
\em{acidtrab} & Indicar se foi acidente de trabalho. 1: sim, 2: não, 9: ignorado & [1, 1]\\
\em{altcausa} & Houve correção ou  alteração da causa do óbito após investigação. 1: sim, 2: não & [1, 1]\\
\em{alteradn} & Se investigação alterou ou corrigiu outro campo da DB, além das causas do óbito. 1: sim, 2: não & [1, 1]\\
\em{alterado} & Se investigação alterou ou corrigiu outro campo da DO, além das causas do óbito. 1: sim, 2: não & [1, 1]\\
\em{alterblocv} & Alteração do bloco V. 1: sim, 2: não & [1, 1]\\
\addlinespace
\em{assistmed} & Assistência médica. 1: sim, 2: não, 9: ignorado & [1, 1]\\
\em{atestado} & CIDs informado no atestado. & [1, 100]\\
\em{atestante} & Indica se o médico que atendeu o paciente. 1: sim, 2: substituto, 3: IML, 4: SVO, 5: outros & [1, 1]\\
\em{baiocor} & Bairro de ocorrência. & [1, 60]\\
\em{baires} & Bairro de residência. & [1, 60]\\
\addlinespace
\em{basica} &  & [1, 8]\\
\em{causabas} & Causa básica da DO. actragrp.CNV & [3, 3]\\
\em{causabas\_d} &  & [1, 4]\\
\em{causabas\_f} &  & [1, 4]\\
\em{causabas\_o} & Causa básica original. CID10CAP.CNV & [3, 3]\\
\addlinespace
\em{causabas\_r} & Causa básica resselecionada. CID10CAC.CNV & [4, 4]\\
\em{causamat} & Causa externa associada a uma causa materna. & [1, 4]\\
\em{cb\_f} &  & [1, 4]\\
\em{cb\_orig} &  & [1, 4]\\
\em{cb\_pre} & Causa básica informada antes da resseleção. CID10CAC.CNV & [4, 4]\\
\addlinespace
\em{cbcb} &  & [1, 4]\\
\em{cepacid} & CEP do endereço do acidente. & [8, 8]\\
\em{cepocor} & CEP do endereço de ocorrência. & [8, 8]\\
\em{cepres} & Código de endereçamento postal. & [8, 8]\\
\em{circobito} & Indicar qual foi a provável circunstância de morte não natural. 1: acidente, 2: suicídio, 3: homicídio, 4: outros, 9: ignorado & [1, 1]\\
\addlinespace
\em{cirurgia} & Cirurgia. 1: sim, 2: não, 9: ignorado & [1, 1]\\
\em{co\_id\_do} &  & [12, 12]\\
\em{codbaiocor} & Código do bairro de ocorrência. & [1, 6]\\
\em{codbaires} & Código do bairro de residência. & [1, 6]\\
\em{codcart} & Código do cartório. & [1, 8]\\
\addlinespace
\em{coddisocor} & Código do distrito de ocorrência. & [1, 8]\\
\em{coddisres} & Código do distrito de residência. & [1, 8]\\
\em{codendacid} & Código de endereço do acidente. & [1, 12]\\
\em{codendocor} & Código do endereço de ocorrência. & [1, 12]\\
\em{codendres} & Código do endereço de residência. & [1, 12]\\
\addlinespace
\em{codestab} & Código de estabelecimento. & [7, 7]\\
\em{codestcart} & Código da UF do cartório. & [2, 2]\\
\em{codestocor} & Código de estabelecimento de ocorrência. & [2, 2]\\
\em{codestres} & Código da UF de residência. & [2, 2]\\
\em{codificado} & Se estiver codificado. S: sim, N: não & [1, 1]\\
\addlinespace
\em{codinst} & Código de configuração da instalação. 1º caractere: nível de instalação (M: municipal, R: regional, E: estadual), 2º e 3º caractere: UF de instalação, 4º ao 9º caractere: código do município de instalação, 10º ao 13º caractere: nº da máquina de instalação & [13, 13]\\
\em{codmuncart} & Código do município do cartório. & [6, 6]\\
\em{codmunnatu} & Código do município de naturalidade do falecido. & [6, 6]\\
\em{codmunocor} & Código do município de ocorrência. & [6, 6]\\
\em{codmunres} & Código do município de residência. & [6, 6]\\
\addlinespace
\em{codpaisres} & Código do país de residência. & [1, 3]\\
\em{codregocor} & Código da região de ocorrência. & [1, 1]\\
\em{codregres} & Código da região de residência. & [1, 1]\\
\em{compara\_cb} & Compara causa básica resselecionada com a informada. IGUAL, DIFER & [5, 5]\\
\em{complacid} & Complemento do endereço onde ocorreu o acidente. & [1, 40]\\
\addlinespace
\em{complocor} & Complemento do endereço de ocorrência. & [1, 40]\\
\em{complres} & Complemento da residência. & [1, 40]\\
\em{comunsvoim} & Código do município do SVO ou do IML. & [6, 6]\\
\em{cond\_cb} &  & [1, 2]\\
\em{cond\_mat} &  & [1, 4]\\
\addlinespace
\em{confcausa} &  & [1, 1]\\
\em{confcidade} &  & [1, 1]\\
\em{confidade} &  & [1, 1]\\
\em{confpeso} &  & [1, 2]\\
\em{contato} & Meio de contato do atestante (telefone, fax, email etc.). & [1, 60]\\
\addlinespace
\em{critica} &  & [1, 1]\\
\em{crm} & Nº do CRM. & [1, 30]\\
\em{descacid} & Descrição do acidente. & [1, 100]\\
\em{difdata} & Diferença entre a data de óbito e data do recebimento original. da DO ([DTOBITO]: [DTRECORIG]). & [1, 3]\\
\em{dsevento} & Descrição sumária do acidente. & [1, 20]\\
\addlinespace
\em{dsexplica} & Descrição da explicação das regras de seleção da causa básica. & [1, 40]\\
\em{dstempo} & Tempo de duração dos CIDs informados. & [1, 20]\\
\em{dtatestado} & Data do atestado. & [10, 10]\\
\em{dtcadastro} & Data do cadastro. & [10, 10]\\
\em{dtcadinf} &  & [10, 10]\\
\addlinespace
\em{dtcadinv} & Data de cadastro da investigação de morte materna. & [10, 10]\\
\em{dtconcaso} & Data de conclusão do caso. & [10, 10]\\
\em{dtconinv} & Data da conclusão da investigação. & [10, 10]\\
\em{dtencomite} & Data do encaminhamento ao comitê. & [10, 10]\\
\em{dtinvestig} & Data da investigação. & [10, 10]\\
\addlinespace
\em{dtnasc} & Data do nascimento. & [10, 10]\\
\em{dtobito} & Data do óbito. & [10, 10]\\
\em{dtrecebim} & Data do recebimento. & [10, 10]\\
\em{dtrecorig} & Data do recebimento original. & [10, 10]\\
\em{dtrecoriga} &  & [10, 10]\\
\addlinespace
\em{dtregcart} & Data do registro do cartório. & [10, 10]\\
\em{dtressele} & Data da resseleção. & [10, 10]\\
\em{endacid} & Endereço do acidente. & [1, 100]\\
\em{endocor} & Endereço de ocorrência. & [1, 100]\\
\em{endres} & Endereço de residência. & [1, 100]\\
\addlinespace
\em{esc} & Escolaridade em anos. 1: nenhuma, 2: de 1 a 3 anos, 3: de 4 a 7 anos, 4: de 8 a 11 anos, 5: 12 anos e mais, 9: ignorado & [1, 1]\\
\em{esc2010} & Escolaridade 2010. 0: sem escolaridade, 1: fundamental I (1ª a 4ª série), 2: fundamental II (5ª a 8ª série), 3: médio (antigo 2º grau), 4: superior incompleto, 5: superior completo, 9: ignorado & [1, 1]\\
\em{escfalagr1} & Escolaridade 2010 agregada. 0: sem escolaridade, 1: fundamental I incompleto, 2: fundamental I completo, 3: fundamental II incompleto, 4: fundamental II completo, 5: ensino médio incompleto, 6: ensino médio completo, 7: superior incompleto, 8: superior completo, 9: ignorado, 10: fundamental I incompleto ou inespecífico, 11: fundamental II incompleto ou inespecífico, 12: ensino médio incompleto ou inespecífico & [1, 2]\\
\em{escfalagr2} & Escolaridade em anos. 1: nenhuma, 2: de 1 a 3 anos, 3: de 4 a 7 anos, 4: de 8 a 11 anos, 5: 12 anos e mais, 9: ignorado & [1, 1]\\
\em{escmae} & Escolaridade em anos. 1: nenhuma, 2: de 1 a 3 anos, 3: de 4 a 7 anos, 4: de 8 a 11 anos, 5: 12 anos e mais, 9: ignorado & [1, 1]\\
\addlinespace
\em{escmae2010} & Escolaridade 2010. 0: sem escolaridade, 1: fundamental I (1ª a 4ª série), 2: fundamental II (5ª a 8ª série), 3: médio (antigo 2º grau), 4: superior incompleto, 5: superior completo, 9: ignorado & [1, 1]\\
\em{escmaeagr1} & Escolaridade 2010 agregada. 0: sem escolaridade, 1: fundamental I incompleto, 2: fundamental I completo, 3: fundamental II incompleto, 4: fundamental II completo, 5: ensino médio incompleto, 6: ensino médio completo, 7: superior incompleto, 8: superior completo, 9: ignorado, 10: fundamental I incompleto ou inespecífico, 11: fundamental II incompleto ou inespecífico, 12: ensino médio incompleto ou inespecífico & [1, 2]\\
\em{escmaeagr2} & Escolaridade em anos. 1: nenhuma, 2: de 1 a 3 anos, 3: de 4 a 7 anos, 4: de 8 a 11 anos, 5: 12 anos e mais, 9: ignorado & [1, 1]\\
\em{estabdescr} &  & [1, 140]\\
\em{estciv} & Situação conjugal. 1: solteiro, 2: casado, 3: viúvo, 4: separado judicialmente/divorciado, 5: união estável, 9: ignorado & [1, 1]\\
\addlinespace
\em{exame} & Exame. 1: sim, 2: não, 9: ignorado & [1, 1]\\
\em{expdifdata} &  & [1, 3]\\
\em{explica\_r} & Explicação das regras de resseleção de causa básica. & [1, 250]\\
\em{fonte} & Indicar a fonte da informação. 1: boletim de ocorrência, 2: hospital, 3: família, 4: outra, 9: ignorado & [1, 1]\\
\em{fonteinv} & Fonte de investigação. 1: comitê de morte materna e/ou infantil, 2: visita domiciliar / entrevista família, 3: estabelecimento de saúde / prontuário, 4: relacionado com outros bancos de dados,  5: SVO, 6: IML, 7: outra fonte, 8: múltiplas fontes, 9: ignorado & [1, 1]\\
\addlinespace
\em{fontes} & Combinado de caracteres conforme o preenchimento dos campos de fontes (FONTENTREV, FONTEAMBUL, FONTEPRONT, FONTESVO, FONTEIML, FONTEPROF). se preenchido caractere S, se o campo estiver vazio caractere X & [6, 6]\\
\em{fontesinf} &  & [7, 7]\\
\em{gestacao} & Faixa de semas de gestação. 1: menos 22 semas, 2: 22 a 27 semas, 3: 28 a 31 semas, 4: 32 a 36 semas, 5: 37 a 41 semas, 6: 42 e + semas & [1, 1]\\
\em{gravidez} & Informar o tipo de gravidez. 1: única, 2: dupla, 3: tripla e mais, 9: ignorada & [1, 1]\\
\em{horaobito} & Horário do óbito. & [1, 5]\\
\addlinespace
\em{id\_paciente} &  & [1, 12]\\
\em{idade} & Idade. O primeiro, de 1 dígito, indica a idade: composto de dois subcampos. o primeiro, de 1 dígito, indica a unidade da idade, conforme a tabela a seguir. o segundo, de dois dígitos, indica a quantidade de unidades: 0: idade menor de 1 hora, o subcampo varia de 01 e 59, 1: hora, o subcampo varia de 01 a 23, 2: dias, o subcampo varia de 01 a 29, 3: meses, o subcampo varia de 01 a 11, 4: anos, o subcampo varia de 00 a 99, 5: anos (mais de 100 anos), o segundo subcampo varia de 0 a 99 & [1, 3]\\
\em{idademae} & Idade da mãe. & [2, 2]\\
\em{linhaa} & CIDs informados linha A da DO. & [1, 40]\\
\em{linhaa\_o} & CIDs informados, originalmente, linha A da DO. & [1, 40]\\
\addlinespace
\em{linhab} & CIDs informados linha B da DO. & [1, 40]\\
\em{linhab\_o} & CIDs informados, originalmente, linha B da DO. & [11, 40]\\
\em{linhac} & CIDs informados linha C da DO. & [5, 40]\\
\em{linhac\_o} & CIDs informados, originalmente, linha C da DO. & [1, 40]\\
\em{linhad} & CIDs informados linha D da DO. & [1, 40]\\
\addlinespace
\em{linhad\_o} & CIDs informados, originalmente, linha D da DO. & [1, 40]\\
\em{linhaii} & CIDs informados parte II da DO. & [1, 40]\\
\em{linhaii\_o} & CIDs informados, originalmente, parte II da DO. & [1, 40]\\
\em{lococor} & Local de ocorrência do óbito. 1: hospital, 2: outros estabelecimentos de saúde, 3: domicílio, 4: via pública, 5: outros, 9: ignorado & [1, 1]\\
\em{medico} & Nome do médico. & [1, 60]\\
\addlinespace
\em{morteparto} & Momento do óbito em relação ao parto. 1: antes, 2: durante, 3: após, 9: ignorado & [1, 1]\\
\em{muncadinv} & Código do município investigador. & [1, 12]\\
\em{natural\_} &  & [1, 6]\\
\em{necropsia} & Confirmação do diagnóstico por necrópsia. 1: sim, 2: não, 9: ignorado & [1, 1]\\
\em{nome} & Nome do falecido. & [1, 80]\\
\addlinespace
\em{nomemae} & Nome da mãe do falecido. & [1, 80]\\
\em{nomepai} & Nome do pai do falecido. & [1, 80]\\
\em{nproc} & Códigos da explicação da não resseleção da causa básica. 1: depende de perguntas, 2: causa externa, 3: procedimento médico, 4: causa básica por CID de paralisia, 5: regra F, 6: CID temporário não pode ser causa básica & [1, 1]\\
\em{nressele} & Descrição da explicação da não resseleção da causa básica. & [1, 140]\\
\em{nudiasinf} &  & [1, 3]\\
\addlinespace
\em{nudiasobco} & Diferença entre a data óbito e a data conclusão da investigação, em dias. & [1, 3]\\
\em{nudiasobin} & Diferença  a data óbito e a data cadastro da investigação, em dias. & [1, 3]\\
\em{numendacid} & Número do endereço do acidente. & [1, 12]\\
\em{numendocor} & Número do endereço de ocorrência. & [1, 12]\\
\em{numerodn} & Número da declaração de nascido vivo. & [1, 16]\\
\addlinespace
\em{numerodo} & Número da DO. & [1, 8]\\
\em{numerodv} & Número do dígito verificador. & [1, 1]\\
\em{numerolote} & Número do lote. & [8, 8]\\
\em{numregcart} & Número do registro do cartório. & [1, 8]\\
\em{numres} & Número da residência. & [1, 12]\\
\addlinespace
\em{numsus} & Número do cartão SUS. & [15, 15]\\
\em{obitoevita} & Óbito poderia ter sido evitado. 1: sim, 2: não, 3: inconcluso & [1, 1]\\
\em{obitograv} & Óbito gravidez. 1: sim, 2: não, 9: ignorado & [1, 1]\\
\em{obitoparto} & Informar como foi a morte em relação ao parto. 1: antes, 2: durante, 3: depois, 9: ignorado & [1, 1]\\
\em{obitopuerp} & Óbito no puerpério. 1: sim, até 42 dias após o parto, 2: sim, de 43 dias a 1 ano, 3: não, 9: ignorado & [1, 1]\\
\addlinespace
\em{ocup} & Ocupação habitual e ramo de atividade. GRUPOCUP.CNV & [3, 3]\\
\em{ocupmae} & Ocupação da mãe. CBO2002.CNV & [6, 6]\\
\em{opor\_do} &  & [1, 8]\\
\em{oportinv\_i} &  & [1, 8]\\
\em{oportinv\_m} &  & [1, 8]\\
\addlinespace
\em{origem} &  & [1, 1]\\
\em{parto} & Informar o tipo de parto. 1: vagil, 2: cesáreo, 9: ignorado & [1, 1]\\
\em{peso} & Peso ao nascer em gramas. & [1, 4]\\
\em{qtdfilmort} & Número de filhos mortos. & [1, 2]\\
\em{qtdfilvivo} & Número de filhos vivos. & [1, 2]\\
\addlinespace
\em{racacor} & Raça. 1: branca, 2: preta, 3: amarela, 4: parda, 5: indígena & [1, 1]\\
\em{regra} &  & [1, 80]\\
\em{retroalim} & Retroalimentação. S: sim, N: não & [1, 1]\\
\em{semagestac} & Semanas de gestação. & [1, 2]\\
\em{seriescfal} & Série escolar do falecido. 0: sem escolaridade, 1: fundamental I incompleto, 2: fundamental I completo, 3: fundamental II incompleto, 4: fundamental II completo, 5: ensino médio incompleto, 6: ensino médio completo, 7: superior incompleto, 8: superior completo, 9: ignorado, 10: fundamental I incompleto ou inespecífico, 11: fundamental II incompleto ou inespecífico, 12: ensino médio incompleto ou inespecífico & [1, 2]\\
\addlinespace
\em{seriescmae} & Série escolar da mãe. 0: sem escolaridade, 1: fundamental I incompleto, 2: fundamental I completo, 3: fundamental II incompleto, 4: fundamental II completo, 5: ensino médio incompleto, 6: ensino médio completo, 7: superior incompleto, 8: superior completo, 9: ignorado, 10: fundamental I incompleto ou inespecífico, 11: fundamental II incompleto ou inespecífico, 12: ensino médio incompleto ou inespecífico & [1, 2]\\
\em{sexo} & Sexo. 1: masculino, 2: feminino, 0,9 ignorado & [1, 1]\\
\em{stcampoalt} & Campos alterados declaração de óbito após a investigação. 1: não, 2: sim & [1, 1]\\
\em{stcodifica} & Status de instalação. S: sim, N: não & [1, 1]\\
\em{stcomitemm} & Se o caso foi encaminhado para o comitê de morte mater. 1: sim, 2: não & [1, 1]\\
\addlinespace
\em{stdoepidem} & Status de do epidemiológica. 1: sim, 0: não & [1, 1]\\
\em{stdonova} & Status de do nova. 1: sim, 0: não & [1, 1]\\
\em{stparecer} & Status se o comitê de morte materna deu parecer. 1: sim, 2. não & [1, 1]\\
\em{stressele} & Status da resseleção. S: sim, N: não & [1, 1]\\
\em{tipobito} & Tipo do óbito. 1: fetal, 2: não fetal & [1, 1]\\
\addlinespace
\em{tp\_analise} &  & [1, 4]\\
\em{tpassina} &  & [1, 2]\\
\em{tpcausapar} & Se as causas do óbito corrigidas expressam o parecer do comitê de morte materna. 1: sim, 2: não, 3: não se aplica o comitê não emitiu parecer ainda, 4: não se aplica a vigilância não teve acesso ao parecer emitido pelo comitê & [1, 1]\\
\em{tpmorteoco} & Informar quando a morte ocorreu. 1: gravidez, 2: no parto, 3: no aborto, 4: até 42 dias após o parto, 5: de 43 dias a 1 ano após o parto, 8: não ocorreu nestes períodos, 9: ignorado & [1, 1]\\
\em{tpnivelinv} & Tipo de nível investigador. E: estadual, R: regional, M: municipal & [1, 1]\\
\addlinespace
\em{tpobitocor} & O óbito ocorreu. 1: durante a gestação, 2: durante abortamento, 3: após abortamento, 4: no parto ou até 1 hora após o parto, 5: no puerpério (até 42 dias do término da gestação), 6: entre o 43º dia e até um ano após o término da gestação, 7: investigação não identificou o momento do óbito, 8: mais de 1 ano após o parto, 9: o óbito não ocorreu em nenhuma das circunstâncias acima mencionadas & [1, 1]\\
\em{tpobitoinf} &  & [1, 2]\\
\em{tppos} & Óbito investigado. S: sim, N: não & [1, 1]\\
\em{tpresginfo} & A investigação permitiu o resgate de alguma causa de óbito não informado, ou a correção de alguma antes informada. 1: não acrescentou nem corrigiu informação, 2: sim, permitiu o resgate de novas informações, 3: sim, permitiu a correção de alguma das causas informadas origilmente & [1, 1]\\
\em{ufinform} &  & [2, 2]\\
\addlinespace
\em{valor} &  & [1, 2]\\
\em{versaoscb} & Versão do seletor de causa básica. VERSCB.CNV & [1, 4]\\
\em{versaosist} & Versão do sistema. VERSAO.CNV & [1, 7]\\
\em{verscbpre} & Versão do SCB da resseleção. VERSCB.CNV & [1, 4]\\
\em{vrsressele} & Versão de resseleção. & [1, 14]\\*
\end{longtable}
\endgroup{}

\newpage

\hypertarget{resultados-numuxe9ricos}{%
\section*{Resultados numéricos}\label{resultados-numuxe9ricos}}
\addcontentsline{toc}{section}{Resultados numéricos}

\hypertarget{resultados-gerais}{%
\subsection*{Resultados gerais}\label{resultados-gerais}}
\addcontentsline{toc}{subsection}{Resultados gerais}

\begingroup\fontsize{10}{12}\selectfont

\begin{longtable}{lccc}
\toprule
Variável & Completude [\%] & Conformidade [\%] & Acurácia [\%]\\
\midrule
\endfirsthead
\multicolumn{4}{@{}l}{\textit{(continued)}}\\
\toprule
Variável & Completude [\%] & Conformidade [\%] & Acurácia [\%]\\
\midrule
\endhead

\endfoot
\bottomrule
\endlastfoot
acidtrab & 0.91 & 100.00 & 100.00\\
altcausa & 1.68 & 100.00 & 100.00\\
alteradn & 0.00 & 0.00 & 0.00\\
alterado & 0.00 & 0.00 & 0.00\\
alterblocv & 0.00 & 0.00 & 0.00\\
\addlinespace
assistmed & 62.65 & 100.00 & 100.00\\
atestado & 99.90 & 100.00 & 100.00\\
atestante & 85.79 & 100.00 & 100.00\\
baiocor & 77.01 & 100.00 & 99.99\\
baires & 56.04 & 100.00 & 99.99\\
\addlinespace
basica & 0.00 & 0.00 & 0.00\\
causabas & 100.00 & 0.00 & 0.00\\
causabas\_d & 5.93 & 100.00 & 100.00\\
causabas\_f & 6.49 & 100.00 & 100.00\\
causabas\_o & 99.96 & 19.19 & 100.00\\
\addlinespace
causabas\_r & 33.96 & 60.13 & 100.00\\
causamat & 0.00 & 100.00 & 100.00\\
cb\_f & 21.01 & 100.00 & 100.00\\
cb\_orig & 0.01 & 100.00 & 100.00\\
cb\_pre & 34.92 & 60.49 & 100.00\\
\addlinespace
cbcb & 7.17 & 100.00 & 100.00\\
cepacid & 0.00 & 0.00 & 0.00\\
cepocor & 0.00 & 0.00 & 0.00\\
cepres & 0.00 & 0.00 & 0.00\\
circobito & 17.16 & 100.00 & 98.47\\
\addlinespace
cirurgia & 19.29 & 100.00 & 94.13\\
co\_id\_do & 100.00 & 100.00 & 100.00\\
codbaiocor & 13.52 & 100.00 & 58.94\\
codbaires & 47.73 & 100.00 & 63.17\\
codcart & 3.97 & 100.00 & 100.00\\
\addlinespace
coddisocor & 0.00 & 0.00 & 0.00\\
coddisres & 0.00 & 0.00 & 0.00\\
codendacid & 0.00 & 0.00 & 0.00\\
codendocor & 0.00 & 0.00 & 0.00\\
codendres & 0.00 & 0.00 & 0.00\\
\addlinespace
codestab & 59.97 & 99.70 & 100.00\\
codestcart & 3.88 & 100.00 & 100.00\\
codestocor & 20.34 & 100.00 & 100.00\\
codestres & 20.34 & 100.00 & 100.00\\
codificado & 100.00 & 100.00 & 100.00\\
\addlinespace
codinst & 100.00 & 100.00 & 100.00\\
codmuncart & 6.26 & 100.00 & 100.00\\
codmunnatu & 49.14 & 100.00 & 97.50\\
codmunocor & 100.00 & 100.00 & 100.00\\
codmunres & 100.00 & 100.00 & 99.99\\
\addlinespace
codpaisres & 99.16 & 100.00 & 100.00\\
codregocor & 69.53 & 100.00 & 100.00\\
codregres & 85.36 & 100.00 & 100.00\\
compara\_cb & 33.96 & 100.00 & 100.00\\
complacid & 0.00 & 0.00 & 0.00\\
\addlinespace
complocor & 0.00 & 0.00 & 0.00\\
complres & 0.00 & 0.00 & 0.00\\
comunsvoim & 16.50 & 100.00 & 100.00\\
cond\_cb & 0.00 & 0.00 & 0.00\\
cond\_mat & 0.01 & 100.00 & 100.00\\
\addlinespace
confcausa & 6.96 & 100.00 & 100.00\\
confcidade & 0.00 & 0.00 & 0.00\\
confidade & 6.96 & 100.00 & 100.00\\
confpeso & 0.00 & 0.00 & 0.00\\
contato & 13.76 & 100.00 & 99.99\\
\addlinespace
critica & 13.45 & 100.00 & 100.00\\
crm & 0.00 & 0.00 & 0.00\\
descacid & 0.00 & 0.00 & 0.00\\
difdata & 86.60 & 100.00 & 100.00\\
dsevento & 1.61 & 100.00 & 100.00\\
\addlinespace
dsexplica & 18.84 & 100.00 & 100.00\\
dstempo & 20.34 & 100.00 & 100.00\\
dtatestado & 95.89 & 100.00 & 100.00\\
dtcadastro & 99.31 & 100.00 & 100.00\\
dtcadinf & 2.65 & 100.00 & 100.00\\
\addlinespace
dtcadinv & 3.49 & 100.00 & 100.00\\
dtconcaso & 0.75 & 100.00 & 100.00\\
dtconinv & 3.45 & 100.00 & 100.00\\
dtencomite & 0.00 & 100.00 & 100.00\\
dtinvestig & 9.13 & 100.00 & 100.00\\
\addlinespace
dtnasc & 99.08 & 100.00 & 100.00\\
dtobito & 100.00 & 100.00 & 100.00\\
dtrecebim & 93.07 & 100.00 & 100.00\\
dtrecorig & 59.13 & 100.00 & 100.00\\
dtrecoriga & 64.37 & 100.00 & 100.00\\
\addlinespace
dtregcart & 4.01 & 100.00 & 100.00\\
dtressele & 34.73 & 100.00 & 100.00\\
endacid & 0.32 & 100.00 & 99.47\\
endocor & 0.00 & 0.00 & 0.00\\
endres & 0.00 & 0.00 & 0.00\\
\addlinespace
esc & 61.08 & 90.66 & 81.45\\
esc2010 & 31.82 & 100.00 & 79.29\\
escfalagr1 & 27.79 & 100.00 & 78.13\\
escfalagr2 & 0.00 & 0.00 & 0.00\\
escmae & 3.97 & 93.61 & 91.73\\
\addlinespace
escmae2010 & 2.18 & 100.00 & 91.66\\
escmaeagr1 & 1.53 & 100.00 & 90.94\\
escmaeagr2 & 0.00 & 0.00 & 0.00\\
estabdescr & 0.00 & 0.00 & 0.00\\
estciv & 77.77 & 100.00 & 91.46\\
\addlinespace
exame & 20.12 & 100.00 & 93.28\\
expdifdata & 7.71 & 100.00 & 99.98\\
explica\_r & 19.87 & 100.00 & 100.00\\
fonte & 6.04 & 100.00 & 79.17\\
fonteinv & 8.66 & 100.00 & 99.66\\
\addlinespace
fontes & 2.84 & 100.00 & 100.00\\
fontesinf & 8.01 & 100.00 & 100.00\\
gestacao & 3.96 & 96.49 & 100.00\\
gravidez & 4.40 & 100.00 & 99.33\\
horaobito & 94.21 & 100.00 & 99.48\\
\addlinespace
id\_paciente & 96.21 & 100.00 & 100.00\\
idade & 99.25 & 100.00 & 100.00\\
idademae & 3.93 & 100.00 & 100.00\\
linhaa & 98.81 & 100.00 & 100.00\\
linhaa\_o & 90.62 & 100.00 & 100.00\\
\addlinespace
linhab & 79.68 & 100.00 & 100.00\\
linhab\_o & 71.88 & 0.09 & 100.00\\
linhac & 47.07 & 100.00 & 100.00\\
linhac\_o & 42.13 & 100.00 & 100.00\\
linhad & 21.73 & 100.00 & 100.00\\
\addlinespace
linhad\_o & 19.48 & 100.00 & 100.00\\
linhaii & 24.76 & 100.00 & 100.00\\
linhaii\_o & 21.82 & 100.00 & 100.00\\
lococor & 100.00 & 100.00 & 99.88\\
medico & 0.00 & 0.00 & 0.00\\
\addlinespace
morteparto & 2.76 & 100.00 & 97.55\\
muncadinv & 0.00 & 0.00 & 0.00\\
natural\_ & 60.70 & 100.00 & 100.00\\
necropsia & 67.12 & 100.00 & 97.16\\
nome & 0.00 & 0.00 & 0.00\\
\addlinespace
nomemae & 0.00 & 0.00 & 0.00\\
nomepai & 0.00 & 0.00 & 0.00\\
nproc & 0.77 & 100.00 & 100.00\\
nressele & 0.11 & 100.00 & 100.00\\
nudiasinf & 0.27 & 100.00 & 100.00\\
\addlinespace
nudiasobco & 1.62 & 100.00 & 100.00\\
nudiasobin & 2.92 & 100.00 & 100.00\\
numendacid & 0.00 & 0.00 & 0.00\\
numendocor & 0.00 & 0.00 & 0.00\\
numerodn & 0.00 & 0.00 & 0.00\\
\addlinespace
numerodo & 100.00 & 100.00 & 100.00\\
numerodv & 78.29 & 100.00 & 81.85\\
numerolote & 100.00 & 100.00 & 100.00\\
numregcart & 3.95 & 100.00 & 99.99\\
numres & 0.00 & 0.00 & 0.00\\
\addlinespace
numsus & 13.90 & 95.35 & 100.00\\
obitoevita & 0.00 & 0.00 & 0.00\\
obitograv & 4.90 & 100.00 & 95.26\\
obitoparto & 4.26 & 100.00 & 97.91\\
obitopuerp & 4.81 & 100.00 & 95.07\\
\addlinespace
ocup & 51.50 & 0.00 & 0.00\\
ocupmae & 3.32 & 99.99 & 100.00\\
opor\_do & 0.00 & 0.00 & 0.00\\
oportinv\_i & 0.00 & 0.00 & 0.00\\
oportinv\_m & 0.00 & 0.00 & 0.00\\
\addlinespace
origem & 72.25 & 100.00 & 100.00\\
parto & 4.35 & 100.00 & 99.02\\
peso & 4.12 & 100.00 & 100.00\\
qtdfilmort & 3.02 & 100.00 & 100.00\\
qtdfilvivo & 3.48 & 100.00 & 100.00\\
\addlinespace
racacor & 80.21 & 100.00 & 100.00\\
regra & 0.00 & 0.00 & 0.00\\
retroalim & 13.45 & 100.00 & 100.00\\
semagestac & 2.23 & 100.00 & 100.00\\
seriescfal & 3.62 & 100.00 & 100.00\\
\addlinespace
seriescmae & 0.50 & 100.00 & 100.00\\
sexo & 100.00 & 100.00 & 99.90\\
stcampoalt & 0.01 & 100.00 & 100.00\\
stcodifica & 99.97 & 100.00 & 100.00\\
stcomitemm & 0.01 & 100.00 & 100.00\\
\addlinespace
stdoepidem & 79.64 & 100.00 & 100.00\\
stdonova & 79.66 & 100.00 & 100.00\\
stparecer & 0.00 & 100.00 & 100.00\\
stressele & 34.73 & 100.00 & 100.00\\
tipobito & 100.00 & 100.00 & 100.00\\
\addlinespace
tp\_analise & 0.00 & 0.00 & 0.00\\
tpassina & 0.00 & 0.00 & 0.00\\
tpcausapar & 0.00 & 100.00 & 100.00\\
tpmorteoco & 3.40 & 100.00 & 94.07\\
tpnivelinv & 2.30 & 100.00 & 100.00\\
\addlinespace
tpobitocor & 4.04 & 100.00 & 100.00\\
tpobitoinf & 0.00 & 0.00 & 0.00\\
tppos & 24.42 & 100.00 & 100.00\\
tpresginfo & 0.10 & 100.00 & 100.00\\
ufinform & 34.97 & 100.00 & 100.00\\
\addlinespace
valor & 0.00 & 0.00 & 0.00\\
versaoscb & 92.75 & 100.00 & 100.00\\
versaosist & 98.33 & 86.02 & 100.00\\
verscbpre & 15.45 & 100.00 & 100.00\\
vrsressele & 34.73 & 100.00 & 100.00\\*
\end{longtable}
\endgroup{}

\hypertarget{resultados-por-ano}{%
\subsection*{Resultados por ano}\label{resultados-por-ano}}
\addcontentsline{toc}{subsection}{Resultados por ano}

\begingroup\fontsize{10}{12}\selectfont

\begin{longtable}{lccc}
\toprule
Ano & Completude [\%] & Conformidade [\%] & Acurácia [\%]\\
\midrule
\endfirsthead
\multicolumn{4}{@{}l}{\textit{(continued)}}\\
\toprule
Ano & Completude [\%] & Conformidade [\%] & Acurácia [\%]\\
\midrule
\endhead

\endfoot
\bottomrule
\endlastfoot
2006 & 100.00 & 93.47 & 99.34\\
2007 & 100.00 & 93.68 & 99.44\\
2008 & 100.00 & 92.97 & 99.18\\
2009 & 100.00 & 92.85 & 99.07\\
2010 & 100.00 & 92.80 & 98.65\\
\addlinespace
2011 & 100.00 & 93.11 & 98.80\\
2012 & 100.00 & 93.08 & 98.64\\
2013 & 100.00 & 93.38 & 98.25\\
2014 & 100.00 & 93.46 & 98.41\\
2015 & 100.00 & 95.11 & 97.94\\
\addlinespace
2016 & 100.00 & 93.88 & 97.61\\
2017 & 100.00 & 92.77 & 97.54\\
2018 & 100.00 & 92.71 & 97.27\\*
\end{longtable}
\endgroup{}

\newpage

\hypertarget{testes-de-inconsistuxeancia}{%
\section*{Testes de inconsistência}\label{testes-de-inconsistuxeancia}}
\addcontentsline{toc}{section}{Testes de inconsistência}

\hypertarget{testes-realizados}{%
\subsection*{Testes realizados}\label{testes-realizados}}
\addcontentsline{toc}{subsection}{Testes realizados}

\begin{itemize}
\tightlist
\item
  \textbf{T1}: A data de óbito não pode ser maior que a data do
  recebimento original
\item
  \textbf{T2}: A data de óbito não pode ser maior que a data do cadastro
\item
  \textbf{T3}: Se óbito no puerpério é 1(Sim, até 42 dias após o parto)
  ou 2(Sim, de 43 dias a 1 ano) ou 3(não), o sexo não ser 1(Masculino)
\item
  \textbf{T4}: A data de óbito não pode ser maior que a data do
  recebimento original
\item
  \textbf{T5}: A data de óbito não pode ser maior que a data do registro
  no cartório
\item
  \textbf{T6}: A data de óbito não pode ser maior que a data do atestado
\item
  \textbf{T7}: A data de óbito não pode ser maior que a data de
  recebimento
\item
  \textbf{T8}: A data de óbito não pode ser maior que a data de
  investigação
\item
  \textbf{T9}: Se o código do estabelecimento existe, o local de
  ocorrência do óbito não pode ser diferente de 1 (hospital) ou 2
  (outros estabelecimentos de saúde)
\item
  \textbf{T10}: Se o óbito do paciente for 2 (ocorreu durante o parto),
  a data de nascimento precisa ser igual a data de óbito
\item
  \textbf{T11}: Se o óbito do paciente for 1 (ocorreu antes do parto), a
  data de nascimento não pode ser menor que a data de óbito
\item
  \textbf{T12}: Se óbito na gravidez é 1 (sim), o sexo não ser 1
  (Masculino)
\item
  \textbf{T13}: Se código do município do SVO ou IML estiver preenchido,
  atestante precisa ser 3 (IML) ou 4 (SVO)
\item
  \textbf{T14}: A escolaridade precisa estar em acordo com a idade, por
  exemplo se a escolaridade for 2 (1 a 3 anos), a idade precisa ser
  superior a 3 anos
\item
  \textbf{T15}: A data de óbito não pode ser menor que a data de
  nascimento
\item
  \textbf{T16}: A escolaridade da mãe precisa estar em acordo com a sua
  idade, por exemplo se a escolaridade for 2 (1 a 3 anos), a idade
  precisa ser superior a 3 anos
\item
  \textbf{T17}: Se o tipo do óbito é 1 (fetal), a data de nascimento
  precisa ser a mesma da data de óbito
\item
  \textbf{T18}: Se o óbito investigado é N (Não) a data de investigação
  não pode estar preenchida.
\end{itemize}

\hypertarget{resultados-obtidos}{%
\subsection*{Resultados obtidos}\label{resultados-obtidos}}
\addcontentsline{toc}{subsection}{Resultados obtidos}

\begingroup\fontsize{10}{12}\selectfont

\begin{longtable}{>{}ccccccccccccccccccc}
\toprule
Ano & T1 & T2 & T3 & T4 & T5 & T6 & T7 & T8 & T9 & T10 & T11 & T12 & T13 & T14 & T15 & T16 & T17 & T18\\
\midrule
\endfirsthead
\multicolumn{19}{@{}l}{\textit{(continued)}}\\
\toprule
Ano & T1 & T2 & T3 & T4 & T5 & T6 & T7 & T8 & T9 & T10 & T11 & T12 & T13 & T14 & T15 & T16 & T17 & T18\\
\midrule
\endhead

\endfoot
\bottomrule
\endlastfoot
\em{2006} & 0 & 8 & 0 & 0 & 0 & 0 & 0 & 0 & 0 & 0 & 0 & 0 & 0 & 0 & 0 & 0 & 0 & 0\\
\em{2007} & 0 & 7 & 0 & 0 & 0 & 0 & 0 & 0 & 0 & 0 & 0 & 0 & 0 & 0 & 0 & 0 & 0 & 0\\
\em{2008} & 0 & 9 & 0 & 0 & 1 & 1 & 0 & 0 & 0 & 0 & 0 & 0 & 0 & 0 & 0 & 0 & 0 & 0\\
\em{2009} & 14 & 10 & 0 & 0 & 0 & 3 & 1 & 0 & 1 & 0 & 0 & 0 & 0 & 0 & 0 & 0 & 0 & 0\\
\em{2010} & 15 & 3 & 1 & 15 & 0 & 5 & 11 & 0 & 0 & 0 & 0 & 0 & 0 & 0 & 0 & 0 & 0 & 0\\
\addlinespace
\em{2011} & 4 & 2 & 0 & 4 & 0 & 2 & 0 & 0 & 0 & 0 & 0 & 0 & 0 & 0 & 0 & 0 & 0 & 0\\
\em{2012} & 15 & 19 & 1 & 15 & 0 & 1 & 0 & 0 & 0 & 0 & 0 & 0 & 0 & 0 & 0 & 0 & 0 & 0\\
\em{2013} & 19 & 37 & 0 & 0 & 0 & 0 & 0 & 0 & 0 & 0 & 0 & 0 & 0 & 0 & 0 & 0 & 0 & 0\\
\em{2014} & 2 & 2 & 3 & 2 & 0 & 2 & 0 & 0 & 0 & 0 & 0 & 0 & 0 & 0 & 0 & 0 & 0 & 0\\
\em{2015} & 5 & 2 & 0 & 5 & 0 & 1 & 0 & 0 & 0 & 0 & 0 & 0 & 0 & 0 & 0 & 0 & 0 & 0\\
\addlinespace
\em{2016} & 4 & 4 & 0 & 4 & 0 & 3 & 0 & 1 & 0 & 0 & 0 & 0 & 0 & 0 & 0 & 0 & 0 & 0\\
\em{2017} & 0 & 2 & 0 & 0 & 0 & 2 & 0 & 0 & 0 & 0 & 0 & 0 & 0 & 0 & 0 & 0 & 0 & 0\\
\em{2018} & 0 & 3 & 0 & 6 & 0 & 0 & 0 & 0 & 0 & 0 & 0 & 0 & 0 & 0 & 0 & 0 & 0 & 0\\*
\end{longtable}
\endgroup{}

\end{document}
